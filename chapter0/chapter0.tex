\documentclass[../main.tex]{subfiles}

\begin{document}
\chapter{Topologie metrischer Räume}\label{chp:normed-spaces}
Das motivierende Thema für dieses Kapitel
ist das Studium von gewöhnlichen Differentialgleichungen.
Bevor wir aber Differentialgleichungen studieren können,
müssen wir den Raum $\mathbb{R}^n$ besser verstehen.
Wir erinnern uns nun an das Studium von Funktionen
in einer Variable.
Dort machen viele Definitionen
vom Absolutbetrag Gebrauch.

\begin{examples}
  \leavevmode
  \begin{enumerate}[(1)]
    \item Konvergenz einer Folge ${(a_{n})}_{n \in \mathbb{N}}$
      von reellen Zahlen,
      in Symbolen
      \[
        \lim_{n \to \infty} a_n = \alpha \in \mathbb{R}
      \]
      heisst Folgendes. Für alle $\varepsilon > 0$ existiert
      $N \in \mathbb{N}$, sodass für alle $n \geq N$
      gilt, dass $|a_n - \alpha| \leq \varepsilon$.
    \item Stetigkeit einer Funktion $f \colon \mathbb{R} \to \mathbb{R}$
      im Punkt $p\in \mathbb{R}$ heisst, dass für alle $\varepsilon > 0$
      ein $\delta > 0$ existiert, so dass
      für alle $q \in \mathbb{R}$ mit $|q - p| \leq \delta$ gilt,
      dass $|f(q) - f(p)| \leq \varepsilon$.
  \end{enumerate}
\end{examples}

Beide dieser sehr zentralen Konzepte machen kritischen Gebrauch
des Absolutbetrags.
In $\mathbb{R}^n$ gibt es aber keinen kanonischen Ersatz für diesen.
Dies motiviert unseren ersten Abschnitt in diesem Kapitel.

\section{Normen auf reellen Vektorräumen}
Die Normaxiome greifen die wichtigsten Eigenschaften des
Absolutbetrags in $\mathbb{R}$ auf und verallgemeinern
diese, so dass wir in allgemeinen reellen Vektorräumen
Konzepte wie Konvergenz und Stetigkeit formalisieren können.

\begin{definition}
  Sei $V$ ein Vektorraum über $\mathbb{R}$.
  Eine Abbildung $\Vert \cdot \Vert \colon V \to \mathbb{R}$
  heisst \emph{Norm} auf $V$, falls folgende Eigenschaften erfüllt werden.
  \begin{enumerate}[(i)]
    \item (Strikte Positivität)
      Für alle $v \in V$ gilt $\Vert v \Vert \geq 0$ und
      $\Vert v \Vert = 0$ genau dann, wenn $v = 0$.
    \item (Homogenität)
      Für alle $v \in V$ und alle $\lambda \in \mathbb{R}$ gilt
      $\Vert \lambda v \Vert = |\lambda| \cdot \Vert v \Vert$.
    \item (Dreiecksungleichung)
      Für alle $v, w \in V$ gilt $\Vert v + w \Vert \leq \Vert v \Vert
      + \Vert w \Vert$.
  \end{enumerate}
\end{definition}

Eigenschaft (ii) hätten wir auch folgendermassen formulieren können:
Die Norm $\Vert \cdot \Vert$ auf einem eindimensionalen Unterraum
von $V$ verhält sich (bis auf Streckung) genau so wie der Absolutbetrag
auf $\mathbb{R}$.

\begin{definition}
  Sei $\Vert \cdot \Vert$ eine Norm auf einem reellen Vektorraum $V$.
  Die Menge
  \[
    B_1 = \left\{v \in V \mid \Vert v \Vert \leq 1\right\} \subset V
  \]
  heisst \emph{Norm-Einheitsball}.
\end{definition}

\begin{examples}
  Sei $V = \mathbb{R}^n$ mit Standardbasis $e_1, \dots, e_n$.
  Einen Vektor $v \in V$ können wir dann ausdrücken durch
  $v = v_1 e_1 + \cdots + v_n e_n$ mit $v_i \in \mathbb{R}$.
  \begin{enumerate}[(1)]
    \item Die \emph{Summennorm} ist die Norm
      \[
        \Vert v \Vert_1 = |v_1| + \cdots + |v_n|.
      \]
      Wir prüfen nun die Normaxiome.
      \begin{enumerate}[(i)]
        \item Dank den Eigenschaften des
          Absolutsbetrag auf $\mathbb{R}$ haben wir sofort
          $\Vert v \Vert_1 \geq 0$
          und $\Vert v \Vert_1 = 0$ genau dann, wenn alle $v_i$ null sind.
        \item Berechne $\Vert \lambda v \Vert_1
          = |\lambda v_1| + \cdots |\lambda v_n| = |\lambda| \cdot
          \Vert v \Vert_1$.
        \item Berechne
          \begin{align*}
            \Vert v + w \Vert_1
            &= |v_1 + w_1| + \cdots + |v_n + w_n| \\
            & \leq |v_1| + |w_1| + \cdots + |v_n| + |w_n| \\
            &= \Vert v \Vert_1 + \Vert w \Vert_1.
          \end{align*}
      \end{enumerate}
    \item Die \emph{Maximumnorm}
      ist die Norm
      \[
        \Vert v \Vert_{\infty}
        = \max \{ |v_1|, \dots |v_n|\}.
      \]
      Wir prüfen wieder die Normaxiome.
      \begin{enumerate}[(i)]
        \item Wir haben $\Vert v \Vert_{\infty} \geq 0$
          und auch $\Vert v \Vert_{\infty} = 0$ genau dann,
          wenn alle $v_i$ null sind.
        \item Es gilt $\Vert \lambda v \Vert_{\infty}
          = \max \left\{|\lambda v_1|, \dots, |\lambda v_n|\right\} = |\lambda|
          \cdot \Vert v \Vert_\infty$.
        \item Berechne
          \begin{align*}
            \Vert v + w \Vert_\infty
            &= \max \left\{|v_1 + w_1|, \dots |v_n + w_n|\right\} \\
            &\leq \max \left\{|v_{1}|, \dots, |v_{n}|\right\}
            + \max \left\{|w_{1}|, \dots, |w_{n}|\right\} \\
            &= \Vert v \Vert_{\infty} + \Vert w \Vert_{\infty},
          \end{align*}
          da jeweils $|v_i + w_i| \leq |v_i| + |w_i| $ gilt.
      \end{enumerate}
    \item die \emph{euklidische Norm} ist die Norm
      \[
        \Vert v \Vert_2 = \sqrt{v_1^2 + \cdots v_n^2}.
      \]
      Auch für diese Norm prüfen wir die Axiome.
      \begin{enumerate}[(i)]
        \item Es gilt $\Vert v \Vert_2 \geq 0$ und
          $\Vert v \Vert_2 = 0$ genau dann, wenn
          $v_1^2 + \cdots + v_n^2 = 0$ gilt,
          was äquivalent dazu ist, dass alle $v_i$ null sind.
        \item Berechne $\Vert \lambda v \Vert_2 = |\lambda | \cdot
          \Vert v \Vert_2$, da $\sqrt{\lambda^2} = |\lambda|$ gilt.
        \item Hier stossen wir zum ersten Mal auf Schwierigkeiten.
          Wir werden das im Lemma unten zeigen.
      \end{enumerate}
    \item
      Folgendes Beispiel rechtfertigt die Notation
      für obige Normen. Sei $p \in \mathbb{R}$ mit $p \geq 1$.
      Die \emph{$p$-Norm} ist
       \[
         \Vert v \Vert_p = \sqrt[p]{|v_1|^p + \cdots + |v_n|^p}.
      \]
      Die Dreiecksungleichung $\Vert v + w \Vert_p \leq
      \Vert v \Vert_p + \Vert w \Vert_p$ heisst
      \emph{Minkowski-Ungleichung}, die aus der
      Konkavität von $\log$ folgt. Siehe hier Abschnitt
      59.3 in~\cite{heuser}.
      Die Normen $\Vert \cdot \Vert_1$ und $\Vert \cdot \Vert_2$
      sind Spezialfälle dieser Familie von Normen.
      Für alle $p \geq 1$ gilt, dass $\Vert v \Vert_{\infty}
      \leq \Vert v \Vert_p \leq \sqrt[p]{n} \Vert v \Vert_{\infty}$.
      Im Grenzwert $p \to \infty$ erhalten wir
      \[
        \lim_{p \to \infty} \Vert v \Vert_p = \Vert v \Vert_{\infty}
      \]
      da $\lim_{p \to \infty} \sqrt[p]{n} = 1$.
  \end{enumerate}
  Die Normbälle der ersten drei Normen im Fall $n = 2$
  sind in Abbildung~\ref{fig:norms}
  zu sehen.
\end{examples}

\begin{figure}[htb]
  \centering
  \begin{minipage}{0.33\textwidth}
    \centering
    \includegraphics{figures/sumnorm}
  \end{minipage}%
  \begin{minipage}{0.33\textwidth}
    \centering
    \includegraphics{figures/maxnorm}
  \end{minipage}%
  \begin{minipage}{0.33\textwidth}
    \centering
    \includegraphics{figures/euclideannorm}
  \end{minipage}%
  \caption{Normbälle der Normen
  $\Vert \cdot \Vert_1,
 \Vert \cdot \Vert_{\infty},
 \Vert \cdot \Vert_2$}%
  \label{fig:norms}
\end{figure}

\subsection*{Normen aus Skalarprodukten}
\begin{definition}
  Ein \emph{Skalarprodukt} auf einem reellen Vektorraum $V$
  ist eine strikt positive, symmetrische, bilineare Abbildung
  $\langle \cdot, \cdot \rangle \colon V \times V \to \mathbb{R}$.
  Das heisst,
  \begin{enumerate}[(i)]
    \item für alle $v \in V$ gilt $\langle v, v \rangle \geq 0$
      und $\langle v, v \rangle = 0$ genau dann,
      wenn $v = 0$,
    \item für alle $v, w \in V$ gilt
      $\langle v, w \rangle = \langle w, v \rangle$,
    \item für alle $u, v, w \in V$ und $t \in \mathbb{R}$ gilt
      $\langle u, v + tw \rangle = \langle u, v \rangle + t
      \langle u, w \rangle$.
  \end{enumerate}
\end{definition}

\begin{lemma*}
  Sei $\langle \cdot, \cdot \rangle$ ein Skalarprodukt auf $V$.
  Dann ist die Abbildung
  \begin{align*}
    \Vert \cdot \Vert \colon V & \to \mathbb{R} \\
    v & \mapsto \sqrt{\langle v, v \rangle}
  \end{align*}
  eine Norm auf $V$.
\end{lemma*}

\begin{proof}
  Wir prüfen die Normaxiome folgendermassen.
  \begin{enumerate}[(i)]
    \item Es gilt $\Vert v \Vert = \sqrt{\langle v, v \rangle} \geq 0$
      und $\Vert v \Vert = 0$ genau dann, wenn
      $\langle v, v \rangle = 0$, das heisst, $v = 0$ gilt.
    \item Berechne $\Vert \lambda v \Vert = \sqrt{\langle \lambda v,
      \lambda v\rangle}
      = \sqrt{\lambda^2 \cdot \langle v, v \rangle} = |\lambda| \cdot
      \Vert v \Vert$.
    \item Seien $v, w \in V$. Wir wollen zeigen, dass
      $\sqrt{\langle v + w, v + w \rangle} \leq
      \sqrt{\langle v, v \rangle} + \sqrt{\langle w, w \rangle}$.
      Da Quadrieren auf $\mathbb{R}_{\geq 0}$ monoton ist,
      reicht es zu zeigen, dass
      \[
      \langle v + w, v+ w \rangle\leq \langle v, v \rangle
      + \langle w, w \rangle + 2 \cdot \sqrt{\langle v, v \rangle} \cdot
      \sqrt{\langle w, w \rangle}.
      \]
      Ausmultiplizieren der linken Seite und Subtraktion der Terme,
      die dann auf beiden Seiten erscheinen liefert, dass
      die Dreiecksungleichung für $\Vert \cdot \Vert$ äquivalent
      zur \emph{Cauchy-Schwarz Ungleichung}
      \[
        \langle v, w \rangle \leq \sqrt{\langle v, v \rangle}
        \cdot \sqrt{\langle w, w \rangle}
      \]
      ist.
      Wir beweisen nun diese.
      Betrachte dazu die Funktion $h \colon \mathbb{R} \to \mathbb{R}$,
      die durch
      \[
        h(t) = \langle v + tw, v + tw \rangle \geq 0
      \]
      definiert ist.
      Es gilt also $h(t) = t^2 \langle w, w \rangle + 2t \langle v, w \rangle
      + \langle v, v \rangle$.
      Dies beschreibt eine Parabel mit Diskriminante
      \[
        D = b^2 - 4ac = 4\langle v, w \rangle^2 - 4\langle w, w \rangle
        \cdot \langle v, v \rangle.
      \]
      Aus $h(t) \geq 0$ für alle $t \in \mathbb{R}$ folgt, dass
      $D \leq 0$ ist.
      Wir schliessen, dass $\langle v, w \rangle^2 \leq \langle w, w \rangle
      \cdot \langle v, v \rangle$ gelten muss.
      \qedhere
  \end{enumerate}
\end{proof}

\begin{examples}
  \leavevmode
  \begin{enumerate}[(1)]
    \item Sei $V = \mathbb{R}^n$ und
      \[
        \langle v, w \rangle = v_1 w_1 + \cdots + v_n w_n
      \]
      das \emph{Standardskalarprodukt}.
      Dann ist
      \[
        \Vert v \Vert = \sqrt{v_1^2 + \cdots + v_n^2} = \Vert v \Vert_2
      \]
      die euklidische Norm.
    \item Sei $V = C[0, 1]$ der Raum von stetigen Funktionen
      $f \colon[0, 1] \to \mathbb{R}$.
      Das ist ein unendlichdimensionaler Vektorraum,
      da die Funktionen $1, x, x^2, \dots$ linear unabhängig sind.
      Für $f, g \in V$ definieren wir
      \[
        \langle f, g \rangle = \int_{0}^{1} f(x) \cdot g(x) \, dx.
      \]
      Die Skalarproduktaxiome sind leicht zu überprüfen.
  \end{enumerate}
\end{examples}

\begin{remark}
  Nicht jede Norm auf $V$ stammt von einem Skalarprodukt.
  In den Übungen wird gezeigt, dass jede von einem Skalarprodukt
  induzierte Norm $\Vert \cdot \Vert$ die
  \emph{Parallelogramm-identität}
  \[
    \Vert v + w \Vert^2 + \Vert v - w \Vert^2
    = 2(\Vert v \Vert^2 + \Vert w \Vert^2)
  \]
  erfüllt, und dass die Normen $\Vert \cdot \Vert_1$
  und $\Vert \cdot \Vert_{\infty}$ diese nicht erfüllen.
\end{remark}

\section{Stetigkeit}\label{sec:continuity}
\begin{definition}
  Seien $V$ und $W$ reelle Vektorräume mit Normen $\Vert \cdot \Vert_V$
  und
  $ \Vert \cdot \Vert_W$.
  Eine Abbildung $f \colon V \to W$ heisst \emph{stetig
  im Punkt} $p \in V$, falls für alle $\varepsilon > 0$
  ein $\delta > 0$ existiert, so dass für alle
  $q \in V$ mit $\Vert q - p \Vert_V \leq \delta$
  folgt, dass
  $\Vert f(q) - f(p) \Vert_W \leq \varepsilon$.
  Wir sagen, dass $f$ \emph{stetig} ist, falls
  $f$ in allen Punkten von $V$ stetig ist.
\end{definition}

\begin{example}
  Im Fall $V = W = \mathbb{R}$ und $\Vert \cdot \Vert = \vert \cdot \vert$
  erhalten wir den üblichen Stetigkeitsbegriff.
\end{example}

Wir untersuchen nun,
wie der Stetigkeitsbegriff von den Normen
auf  $V$ und $W$ abhängt.
Betrachte zunächst folgenden günstigen Fall.
Sei $V = W = \mathbb{R}$. Sei $\Vert \cdot \Vert$
eine Norm auf $\mathbb{R}$.
Für alle $x \in \mathbb{R}$ gilt dann
\[
  \Vert x \Vert = \Vert x \cdot 1 \Vert = |x| \cdot \Vert 1 \Vert.
\]
Sei nun $f \colon \mathbb{R} \to \mathbb{R}$
bezüglich der Norm $| \cdot |$ im Punkt $p \in \mathbb{R}$ stetig.
Wir zeigen nun, dass $f$ auch bezüglich der Norm $\Vert \cdot \Vert$
im Punkt $p$ stetig ist.
Wähle $\delta > 0$ so, dass für alle $q \in \mathbb{R}$
mit $|q - p| \leq \delta/\Vert 1 \Vert$ gilt,
dass $|f(q) - f(p)| \leq \varepsilon/\Vert 1 \Vert$.
Das geht, da $\Vert 1 \Vert > 0$ aus $1 \neq 0$ folgt.
Sei nun $q \in \mathbb{R}$ mit $\Vert q - p \Vert \leq \delta$.
Dann gilt $|q - p| \leq \delta / \Vert 1 \Vert$.
Wir schliessen, dass $|f(q) - f(p)| \leq \varepsilon/\Vert 1 \Vert$,
also
$\Vert f(q) - f(p) \Vert = |f(q) - f(p)| \cdot \Vert 1 \Vert \leq \varepsilon$.
Insgesamt hängt also der Stetigkeitsbegriff auf $\mathbb{R}$ nicht von
der Norm auf $\mathbb{R}$ ab.

Diese Aussage ist jedoch in allgemeinen Vektorräumen falsch.
Sei $\mathbb{R}^{\infty}$ der Raum aller Folgen
$(x_{1}, x_{2}, \dots) \in \mathbb{R}^{\mathbb{N}}$
für welche $N > 0$ existiert, so dass für alle $n \geq N$
gilt, dass $x_n = 0$.
Betrachte auf $\mathbb{R}^{\infty}$ die beiden Normen
\begin{align*}
  \Vert v \Vert_2 & = \sqrt{v_1^2 + v_2^2 + \cdots} \\
  \Vert v \Vert_{\infty} &= \max \{|v_1|, |v_2|, \dots\}.
\end{align*}
Es gilt für alle $v \in \mathbb{R}^{\infty}$,
dass $\Vert v \Vert_{\infty} \leq \Vert v \Vert_2$.
Sei $V = (\mathbb{R}^{\infty}, \Vert \cdot \Vert_2)$
und $W = (\mathbb{R}^{\infty}, \Vert \cdot \Vert_{\infty})$.
Die Identitätsabbildung
\begin{align*}
  \text{Id} \colon \mathbb{R}^{\infty} & \to \mathbb{R}^{\infty} \\
  v & \mapsto v
\end{align*}
kann auf vier Arten als Abbildung zwischen normierten Räumen
betrachtet werden, nämlich
\begin{itemize}
  \item $f_1 \colon V \to V$,
  \item $f_2 \colon W \to W$,
  \item $f_3 \colon V \to W$,
  \item $f_4 \colon W \to V$.
\end{itemize}
Wir bemerken, dass $f_1$, $f_2$ und $f_3$ stetig
sind, indem wir $\delta = \varepsilon$ wählen.
Aber $f_4$ ist nicht stetig!
Betrachte dazu die Folge von Punkten
$p_n = (1/n, 1/n, \dots, 1/n, 0, \dots) \in \mathbb{R}^{\infty}$,
wobei das Glied $1/n$ gerade $n^2$ mal vorkommt.
Es gilt $\Vert p_n \Vert_{\infty} = 1/n$
und $\Vert p_n \Vert_{2} = 1$.
Insbesondere gilt
\[
  \Vert f_4(p_n) - f_4(0) \Vert_2 = \Vert p_n \Vert_2 = 1
\]
und $\Vert p_n \Vert_{\infty} = 1/n$,
also ist $f_4$ im Punkt $p = 0$ nicht stetig.

Zusammengefasst erkennen wir, dass auf $\mathbb{R}^{\infty}$
das Verhältnis der Normen $\Vert \cdot \Vert_2$
und $\Vert \cdot \Vert_{\infty}$ beliebig hohe Werte annimmt.
Im Sinne folgender Definition bedeutet das, dass in
$\mathbb{R}^{\infty}$ diese beiden normen nicht äquivalent sind.

\begin{definition}
  Zwei Normen $\Vert \cdot \Vert$, $\widetilde{\Vert \cdot \Vert}$
  heissen \emph{äquivalent},
  falls es Konstanten $0 < c \leq C$ gibt,
  so dass für alle $v \in V \setminus \{0\}$ gilt, dass
  \[
    c \leq \frac{\widetilde{\Vert v \Vert}}{\Vert v \Vert} \leq C.
  \]
\end{definition}

Bildlich können wir uns die Äquivalenz von Normen so vorstellen,
dass der $\widetilde{\Vert \cdot \Vert}$-Normball
sowohl eine skalierte Version vom $\Vert \cdot \Vert$-Normball
enthält, wie auch in einer skalierten Version
vom $\Vert \cdot \Vert$-Normball enthalten ist.

\begin{examples}
  \leavevmode
  \begin{enumerate}[(1)]
    \item Sei $\Vert \cdot \Vert$ eine Norm auf $\mathbb{R}$.
      Dann gilt für alle $x \in \mathbb{R}$, dass
      \(
        \Vert x \Vert = \Vert 1 \Vert \cdot |x|,
      \)
      also sind $\Vert \cdot \Vert$ und $| \cdot |$ äquivalent.
    \item
      Für alle $v \in \mathbb{R}^n$ gilt, dass
      \[
        \Vert v \Vert_{\infty} \leq \Vert v \Vert_2
        \leq \sqrt{n} \cdot \Vert v \Vert_{\infty}.
      \]
      also sind die beiden Normen
      $\Vert \cdot \Vert_{\infty}$ und $\Vert \cdot \Vert_2$
      äquivalent auf $\mathbb{R}^{n}$.
    \item Auf $\mathbb{R}^{\infty}$ sind die Normen
      $\Vert \cdot \Vert_{\infty}$ und $\Vert \cdot \Vert_{2}$
      nicht äquivalent.
  \end{enumerate}
\end{examples}

\begin{remark}
  Die Äquivalenz von Normen ist eine Äquivalenzrelation.
  Die Symmetrie und die Reflexivität sind leicht zu überprüfen.
  Für die Transitivität bemerke, dass
  aus $a \leq x/y \leq A$ und $b \leq y/z \leq B$ folgt,
  dass $ab \leq x/z \leq AB$.
\end{remark}

\begin{theorem*}
  Auf $\mathbb{R}^n$ sind alle Normen äquivalent.
\end{theorem*}

\begin{proof}
  Es reicht zu zeigen, dass jede Norm
  $\Vert \cdot \Vert$ auf $\mathbb{R}^{n}$
  zur euklidischen Norm $\Vert \cdot \Vert_2$
  äquivalent ist.
  In einem ersten Schritt zeigen wir, dass
  eine Konstante $C > 0$ existiert, so dass
  $\Vert \cdot \Vert / \Vert \cdot \Vert_2 \leq C$
  gilt.
  Sei $v = v_1 e_1 +\cdots + v_n e_n$ mit $v_i \in \mathbb{R}$.
  Schätze ab, dass
  \begin{align*}
    \Vert v \Vert
    &\leq \Vert v_1 e_1 \Vert + \cdots + \Vert v_n e_n \Vert \\
    &= |v_1| \cdot \Vert e_1 \Vert + \cdots + |v_n| \cdot \Vert e_n \Vert \\
    &\leq \Vert v \Vert_2 \cdot (\Vert e_1 \Vert + \cdots + \Vert e_n \Vert).
  \end{align*}
  Wir erhalten das gewünschte Resultat, indem wir
  $C = \Vert e_1 \Vert + \cdots + \Vert e_n \Vert$ setzen,
  dann gilt nämlich $\Vert v \Vert \leq C \cdot \Vert v \Vert_2$.

  Wir suchen nun im zweiten Schritt
  eine Konstante $c > 0$, für die $\Vert \cdot \Vert / \Vert \cdot \Vert_2
  \geq c$ gilt.
  Nimm widerspruchsweise an, dass das Verhältnis $\Vert \cdot \Vert /
  \Vert \cdot \Vert_2$ beliebig kleine Werte annimmt.
  Dann gibt es eine Folge von Vektoren
  $v_k \in \mathbb{R}^n \setminus \{0\}$ mit
  $\Vert v_k \Vert / \Vert v_k \Vert_2 \leq 1/k$.
  Für die normierten Vektoren $w_k = v_k / \Vert v_k \Vert_2
  \in \mathbb{R}^n \setminus \{0\}$
  gilt dann, dass $\Vert w_k \Vert_2 = 1$
  und $\Vert w_k \Vert \leq 1/k$.
  Somit liegen alle Koordinaten von $w_k$
  im Intervall $[-1, 1]$.
  Also existiert eine Teilfolge
  der $w_k$, so dass die erste Koordinatenfolge
  mit Grenzwert in $[-1, 1]$ konvergiert.
  Wir iterieren dieses Verfahren $n$-mal,
  und erhalten eine Teilfolge $w_{k_i}$ der $w_k$,
  welche in allen Koordinaten konvergiert.
  Setze $w = \lim_{i \to \infty} w_{k_i}$.
  Wir behaupten nun, dass $\Vert w \Vert_2 = 1$
  und $\Vert w \Vert = 0$ gilt, was
  natürlich den Widerspruch $w = 0$ impliziert.
  Bemerke dazu, dass für alle $i \in \mathbb{N}$
  gilt, dass
  \begin{align*}
     1 - \Vert w - w_{k_i} \Vert_2
     &= \Vert w_{k_i} \Vert_2 - \Vert w - w_{k_i} \Vert_2  \\
     &\leq \Vert w \Vert_2 \\
     &\leq \Vert w - w_{k_i} \Vert_2 + \Vert w_{k_i} \Vert_2 \\
    &= 1 + \Vert w - w_{k_i} \Vert_2.
  \end{align*}
  Da die $w_{k_i}$ koordinatenweise gegen $w$ konvergieren, folgt, dass
  \[
    \lim_{i \to \infty} \Vert w - w_{k_i}\Vert_2 = 0
  \]
  gilt. Schätze weiterhin ab, dass
  \[
     \Vert w \Vert
     \leq \Vert w_{k_i} \Vert + \Vert w - w_{k_i} \Vert ,
  \]
  also existiert nach Schritt 1 eine Konstante
  $C > 0$, sodass $\Vert w \Vert \leq 1/k_i + C \cdot
  \Vert w - w_{k_i} \Vert_2$
  gilt, woraus im Grenzwert $i \to \infty$ folgt,
  dass $\Vert w \Vert = 0$ gilt.
\end{proof}

\begin{corollary}\label{cor:continuity-independent}
  Der Stetigkeitsbegriff für Abbildungen
  $f \colon \mathbb{R}^n \to \mathbb{R}^m$
  hängt nicht von den Normen auf $\mathbb{R}^n$
  und $\mathbb{R}^m$ ab.
\end{corollary}

Der Beweis hier lässt sich durch einfache Ajustierung
der $\delta$ und $\varepsilon$ führen.
Siehe dazu unseren Beweis für den Fakt,
dass alle Normen auf $\mathbb{R}$ zum
selben Stetigkeitsbegriff führen.

\subsection*{Lineare Abbildungen}
\begin{question}
Seien $V, W$ normierte Vektorräume und
$A \colon V \to W$ linear.
Ist $A$ dann stetig?
\end{question}

Die Antwort ist im allgemeinen nein, was unsere
Abbildung $f_4$ oben zeigt. Wir untersuchen
nun noch ein weiteres Beispiel.

\begin{example}
  Sei $V = \mathbb{R}^{\infty}$ mit der
  Maximumsnorm $\Vert \cdot \Vert_{\infty}$.
  Definiere $A \colon \mathbb{R}^{\infty} \to \mathbb{R}^{\infty}$
  als die lineare Erweiterung der Abbildung $e_k \mapsto ke_k$.
  Setze  $q_k = e_k/k \in \mathbb{R}^{\infty}$.
  Es gilt, dass
  \(
    \Vert q_k - 0 \Vert_{\infty} = \Vert q_k \Vert_{\infty} = 1/k
  \)
  und $\Vert A(q_k) - A(0) \Vert_{\infty} = \Vert e_k - 0 \Vert_{\infty}
  = 1$. Somit ist $A$ im Nullpunkt (und auch in allen anderen Punkten)
  nicht stetig.
\end{example}

\begin{definition}
  Seien $\Vert \cdot \Vert_V$, $\Vert \cdot \Vert_W$
  Normen auf normierten Vektorräumen $V, W$.
  Sei $A \colon V \to W$ linear.
  Die \emph{Operatornorm} von $A$ ist
  \[
    \Vert A \Vert_{\text{op}} = \sup
    \left\{\frac{\Vert Av \Vert_W}{\Vert v \Vert_V} \mid v \in
    V \setminus \{0\}\right\} \in \mathbb{R} \cup \{\infty\}.
  \]
\end{definition}

\begin{examples}
  \leavevmode
  \begin{enumerate}[(1)]
    \item Betrachte die Abbildung
      \begin{align*}
        A \colon \mathbb{R}^{\infty} & \to \mathbb{R}^{\infty} \\
        e_k & \mapsto ke_k
      \end{align*}
      auf $\mathbb{R}^{\infty}$ mit der Norm $\Vert \cdot \Vert_{\infty}$.
      Dann gilt $\Vert A \Vert_{\text{op}} = \infty$.
    \item Sei $A \colon \mathbb{R}^2 \to \mathbb{R}^2$
      auf $\mathbb{R}^{2}$ mit der euklidischen Norm
      $\Vert \cdot \Vert_2$ die Abbildung, die durch
      die Matrix
      \[
        A =
        \begin{pmatrix}
          a & b \\ c & d
        \end{pmatrix}
      \]
      gegeben ist.
      Bemerke folgende Eigenschaften.
      \begin{enumerate}[(i)]
        \item $A = 0$ genau dann, wenn $\Vert A \Vert_\text{op} = 0$.
        \item Betrachte
          \[
            A =
            \begin{pmatrix}
              0 & 1 \\ 0 & 0
            \end{pmatrix}.
          \]
          Sei $v = (x, y)$.
          Es gilt dann $Av = (y, 0)$, also gilt
          \[
            \frac{\Vert Av \Vert_2}{\Vert v \Vert_2} = \frac{y}{\sqrt{x^2 + y^2} } \leq 1.
          \]
          Für $x = 0$ erhalten wir $1$ für diesen Quotienten,
          also gilt $\Vert A \Vert_{\text{op}} = 1$.
        \item Betrachte die Matrix
          \[
            A =
            \begin{pmatrix}
              1 & 0 \\ 0 & d
            \end{pmatrix}
          \]
          für $d \geq 1$.
          Um $\Vert A \Vert_{\text{op}}$ zu bestimmen
          reicht es, Vektoren der Form $v = (1, 0)$
          und $v = (x, 1)$ zu betrachten.
          Es gilt $A(1, 0) = (1, 0)$,
          also ist $\Vert A \Vert_{\text{op}} \geq 1$.
          Weiter ist
          $A(x, 1) = (x, d)$.
          Wir schliessen, dass für $v = (x, 1)$ gilt, dass
          \[
            \frac{\Vert Av \Vert_2}{\Vert v \Vert_2}
            = \frac{\sqrt{x^2 + d^2}}{\sqrt{x^2 + 1^2}} \leq d.
          \]
          Aber $d$ wird im Fall $x = 0$ angenommen,
          also folgt $\Vert A \Vert_{\text{op}} = d$.
      \end{enumerate}
  \end{enumerate}
\end{examples}

\begin{proposition}
  Sei $A \colon V \to W$ eine lineare Abbildung zwischen
  normierten Vektorräumen.
  Dann ist $A$ stetig, genau dann, wenn
  $\Vert A \Vert_\text{op} < \infty$ gilt.
\end{proposition}

\begin{proof}
Seien $p \in V$ und $\varepsilon > 0$
vorgegeben.
Setze
\[
  \delta = \frac{\varepsilon}{\Vert A \Vert_{\text{op}} + 1}.
\]
Sei nun $q \in V$ mit $\Vert q - p \Vert_V \leq \delta$.
Dann gilt
\begin{align*}
  \Vert A(q) - A(p) \Vert_W
  & = \Vert A(q- p)\Vert_W\\
  &\leq \Vert A \Vert_\text{op} \cdot \Vert q - p \Vert_V \\
  & \leq \Vert A \Vert_{\text{op}} \cdot
  \frac{\varepsilon}{\Vert A \Vert_{\text{op}} + 1} \\
  &\leq \varepsilon.
\end{align*}
Somit ist $A$ stetig im Punkt $p \in V$.
Für die Umkehrung, siehe Serie 2.
\end{proof}

\begin{specialcase}
  Sei $A \colon \mathbb{R}^n \to \mathbb{R}^m$ linear.
  Schätze $\Vert A \Vert_{\text{op}}$ bezüglich
  der Maximumsnorm $\Vert \cdot \Vert_{\infty}$ auf
  $\mathbb{R}^n$ und $\mathbb{R}^m$ wie folgt ab.
  Schreibe
  \[
    A =
    \begin{pmatrix}
      a_{11} & \cdots & a_{1n} \\
      \vdots & \ddots & \vdots \\
      a_{m1} & \cdots & a_{mn}
    \end{pmatrix}
  \]
  als Matrix, das heisst $A(e_i) = \sum_{j=1}^{m} a_{ij}e_j$.
  Setze $a = \max \left\{|a_{ij}| \mid
  1 \leq i \leq m, 1 \leq j \leq n\right\}$.
  Für $v \in \mathbb{R}^{n}$ gilt dann
  \(
    \Vert Av \Vert_\infty \leq n \cdot a \cdot \Vert v \Vert_{\infty}.
  \)
  Insbesondere ist $\Vert A \Vert_{\text{op}} \leq n \cdot a$ endlich.
  Wir schliessen, dass $A \colon \mathbb{R}^n \to \mathbb{R}^m$ stetig ist.
\end{specialcase}

\begin{corollary}
  Lineare Abbildungen $A \colon \mathbb{R}^n \to \mathbb{R}^m$ sind stetig.
\end{corollary}

\begin{proof}
  Wir haben soeben gezeigt, dass $A$ bezüglich der Maximumsnorm stetig ist.
  Aus Korollar~\ref{cor:continuity-independent} folgt, dass
  $A$ bezüglich jeder Norm stetig ist.
\end{proof}

\section{Der Fixpunktsatz von Banach}
\begin{motivation}
  Sei $V$ ein normierter reeller Vektorraum,
  und sei $A \colon V \to V$ linear und \emph{kontrahierend},
  das heisst, es gilt
  $\Vert A \Vert_{\text{op}} < 1$.
  Wir wollen die Abbildung $\text{Id} - A \colon V \to V$ wie folgt
  invertieren:
  \[
    {(\text{Id} - A)}^{-1} = \textup{Id} + A + A^2 + A^3 + \cdots
  \]
  Als Analogie dazu erinnern wir uns an den Fakt, dass
  für $q \in \mathbb{R}$ mit $|q| < 1$ gilt, dass
  \[
    \frac{1}{1-q} = 1 + q + q^2 + \cdots.
  \]

  Für alle $Y \in V$ suchen wir ein eindeutiges $X \in V$
  mit $(\text{Id} - A)(X) = Y$, also
  $X - A(X) = Y$.
  Definiere
  \begin{align*}
    f \colon V & \to V \\
    X & \mapsto A(X) + Y.
  \end{align*}
  Gesucht ist also eine (offenbar eindeutige) Lösung
  der Fixpunkgleichung $f(X) = X$.
\end{motivation}

\begin{remark}
  Die Bedingung $\Vert A \Vert_{\text{op}} < 1$ ist stärker als die
  Bedingung $\Vert A v \Vert_V < \Vert v \Vert_V$ für alle $v
  \in V \setminus \{0\}$.
  Betrachte dazu $V = \mathbb{R}^{\infty}$ mit der Maximumnorm
  $\Vert \cdot \Vert_{\infty}$.
  Definiere $A \colon \mathbb{R}^{\infty} \to \mathbb{R}^{\infty}$
  als die lineare Erweiterung der Abbildung
  \[
    e_k \mapsto \frac{k}{k+1}e_k.
  \]
  Wir stellen fest, dass $\Vert A v \Vert_{\infty} < \Vert v \Vert_{\infty}$
  für alle $v \in \mathbb{R}^{\infty} \setminus \{0\}$ gilt,
  aber $\Vert A \Vert_{\text{op}} = 1$, da
  \[
    \sup \left\{\frac{k}{k+1} \mid k \in \mathbb{N}\right\} = 1
  \]
  gilt. Somit ist $A$ nicht kontrahierend.

  Wir stossen auch auf ein Problem beim Invertieren von
  \begin{align*}
    \text{Id} - A \colon \mathbb{R}^{\infty} & \to \mathbb{R}^{\infty} \\
    e_k & \mapsto \left( 1 - \frac{k}{k+1} \right)e_k
    = \frac{1}{k+1}e_k.
  \end{align*}
  Die inverse Abbildung
  \begin{align*}
    {(\text{Id} - A)}^{-1} \colon \mathbb{R}^{\infty} & \to \mathbb{R}^{\infty} \\
    e_k & \mapsto (k+1) e_k
  \end{align*}
  ist also nicht stetig, da $\Vert {(\text{Id} - A)}^{-1} \Vert_{\text{op}}
  = +\infty$.
\end{remark}

\subsection*{Metrische Räume}
\begin{definition}
  Sei $X$ eine Menge. Eine \emph{Metrik} auf $X$ ist eine
  Abbildung $d \colon X \times X \to \mathbb{R}$
  mit folgenden Eigenschaften.
  \begin{enumerate}[(i)]
    \item \emph{Positivität}: Für alle $p, q \in X$ gilt $d(p, q) \geq 0$, und
      $d(p, q) = 0$ genau dann, wenn $p = q$.
    \item \emph{Symmetrie}. Für alle $p, q \in X$ gilt $d(q, p) = d(p, q)$.
    \item \emph{Dreiecksungleichung}.
      Für alle  $a, b, c \in X$ gilt $d(a, c) \leq d(a, b) +
      d(b, c)$.
  \end{enumerate}
\end{definition}

\begin{examples}
  \leavevmode
  \begin{enumerate}[(1)]
    \item Sei $V$ ein reeller Vektorraum mit Norm $\Vert \cdot \Vert$.
      Definiere
      \begin{align*}
        d \colon V \times V & \to \mathbb{R} \\
        (p, q) & \mapsto \Vert q - p \Vert.
      \end{align*}
      Das erste Axiom für Metrische Räume folgt direkt aus dem
      ersten Axiom für normierte Räume.
      Zur Symmetrie bemerke, dass
      \[
        d(p, q) = \Vert q - p  \Vert = \Vert (-1)(p - q) \Vert
        = |-1| \cdot \Vert p - q \Vert = \Vert p - q \Vert
        = d(q, p).
      \]
      Die Dreiecksungleichung erfordert keinen Trick.
    \item Sei $X$ beliebig. Definiere auf $X$
      die \emph{diskrete Metrik} wie folgt:
      \[
        d (p, q)
        =
        \begin{cases}
          0 & q = p \\
          1 & q \neq p
        \end{cases}.
      \]
      Das einzige interessante Axiom ist die Dreiecksungleichung.
      Falls $d(a, c) = 0$ gilt, dann gilt sowieso
      $d(a, c) \leq d(a, b) + d(b, c)$.
      Der einzige Fall, den wir ausschliessen müssen, ist also
      $d(a, c) = 1$ und $d(a, b) = d(b, c) = 0$.
      Aber aus $d(a, b) = 0$ folgt $a = b$, und
      aus $d(b, c) = 0$ folgt $b = c$ und somit
      auch $a = c$, was $d(a, c) = 0$ impliziert. Dies ist ein
      Widerspruch.
    \item Folgende Metrik ist enorm wichtig in der Zahlentheorie.
      Wir werden sie aber bloss in den Übungen antreffen.
      Sei $p \in \mathbb{N}$ eine Primzahl.
      Definiere die \emph{$p$-adische Metrik} auf $\mathbb{Z}$
      wie folgt:
      \[
        d_p(n, m) =
        \begin{cases}
          0 & n = m \\
          p^{-x} & n \neq m,
        \end{cases}
      \]
      wobei $x \in \mathbb{N}$ maximal mit der Eigenschaft
      ist, dass
      $p^x$ die Zahl $n -m $ teilt.
      Insbesondere gilt $d_p(n, n + 1) = 1 = p^{-0}$,
      da in diesem Fall $x = 0$ gilt.
      Weiter ist $d_p(0, p^n) = p^{-n}$, ist also klein für
      grosse $n$.
  \end{enumerate}
\end{examples}

%TODO remove this
\newpage

\begin{definitions}
  Sei $X$ ein metrischer Raum mit Metrik $d$.
  \begin{itemize}
    \item Eine Folge ${(a_{n})}_{n \in \mathbb{N}}$ in
      $X$ heisst \emph{konvergent} mit Grenzwert
      $a \in X$, falls für alle $\varepsilon$
      ein $N \in \mathbb{N}$ existiert,
      so dass für alle $n \in \mathbb{N}$ mit $n \geq N$
      gilt, dass $d(a_n, a) \leq \varepsilon$.
    \item Eine Folge ${(a_{n})}_{n \in \mathbb{N}}$ in $X$
      heisst \emph{Cauchyfolge}, falls für alle $\varepsilon > 0$
      ein $N \in \mathbb{N}$ existiert, so dass für
      alle $n, m \geq N$ gilt, dass
      $d(a_n, a_m) \leq \varepsilon$.
    \item Der metrische Raum $X$ heisst \emph{vollständig},
      falls alle Cauchyfolgen in $X$ konvergieren.
  \end{itemize}
  Ein vollständiger normierter Vektorraum heisst \emph{Banachraum}.
\end{definitions}

\begin{examples}
  \leavevmode
  \begin{enumerate}[(1)]
    \item Der metrische Raum $\mathbb{R}$ mit $d(p, q) = |p - q|$
      ist vollständig.
    \item
      \begin{enumerate}[(a)]
        \item Der metrische Raum $X = (0, 1) \subset \mathbb{R}$
          mit der von $\mathbb{R}$ induzierten
          Metrik $d(p, q) = |q - p|$ ist unvollständig.
        \item Ebenso ist $X = \mathbb{Q} \subset \mathbb{R}$ mit
           der von $\mathbb{R}$ induzierten Metrik unvollständig.
      \end{enumerate}
    \item Sei $X = \mathbb{Z}$ mit der $p$-adischen Metrik $d_p$.
      Betrachte die Folge $a_n = p^n$.
      Dies ist eine Cauchyfolge mit Grenzwert $0 \in \mathbb{Z}$.
      Man könnte nun vermuten, dass $\mathbb{Z}$ mit dieser
      Metrik vollständig ist. Das ist jedoch nicht der Fall.
      Betrachte dazu die Folge $b_n = 1 + p + p^2 + \cdots + p^n$.
      Dies ist eine Cauchyfolge, leider aber ohne Grenzwert
      in $\mathbb{Z}$.
      Das heisst, dass $\mathbb{Z}$ mit der $p$-adischen Metrik
      unvollständig ist. Siehe dazu auch die Übungen.
  \end{enumerate}
\end{examples}

\begin{definition}
  Sei $(X, d)$ ein metrischer Raum.
  Eine Abbildung $f \colon X \to X$ heisst \emph{kontrahierend},
  falls es eine positive Konstante $k < 1$ gibt,
  so dass für alle $p, q \in X$ gilt,
  dass $d(f(p), f(q)) \leq k \cdot d(p, q)$.
\end{definition}

Unser Ziel in diesem Abschnitt ist es,
folgenden wichtigen Satz des polnischen
Mathematikers Stefan Banach (1892--1945)
zu beweisen.

\begin{theorem*}[Fixpunktsatz von Banach 1922, Heuser 111.11]
  Sei $(X, d)$ ein vollständiger nicht-leerer metrischer Raum.
  Dann hat jede kontrahierende Abbildung
  $f \colon X \to X$ einen eindeutigen Fixpunkt,
  das heisst es existiert genau ein Punkt $p \in X$
  mit $f(p) = p$.
\end{theorem*}

\begin{remarks}
  \leavevmode
  \begin{enumerate}[(1)]
    \item Es ist wichtig, dass $k < 1$ gilt.
      Betrachte zum Beispiel folgende Funktionen.
      \begin{enumerate}[(a)]
        \item Betrachte die Abbildung
          \begin{align*}
            f \colon \mathbb{R} & \to \mathbb{R} \\
            x & \mapsto x + 1.
          \end{align*}
          Dann gilt $|f(y) - f(x)| = 1 \cdot |y - x|$,
          aber $f$ hat keinen Fixpunkt.
        \item Die Identität
          \begin{align*}
            f \colon \mathbb{R} & \to \mathbb{R} \\
            x & \mapsto x
          \end{align*}
          hat unendlich viele Fixpunkte.
      \end{enumerate}
    \item Auch die Vollständigkeit ist wichtig.
      Betrachte hier zum Beispiel die Funktion
      \begin{align*}
        f \colon \mathbb{R} \setminus \{0\} & \to \mathbb{R} \setminus \{0\} \\
        x & \mapsto x/2
      \end{align*}
      Die Abbildung $f$ ist kontrahierend mit $k = 1/2$,
      hat aber keinen Fixpunkt in $\mathbb{R} \setminus \{0\}$.
      In den Übungen werden wir ein weniger künstliches Beispiel
      untersuchen.
    \item
      Kontrahierende Abbildungen $f \colon X \to X$
      sind wie alle Lipschitz-stetigen Abbildungen
      $g \colon X \to X$ gleichmässig stetig:
      Sei $L \geq 0$ so, dass für alle
      $p, q \in X$
      gilt, dass
      $d(g(p), g(q)) \leq L \cdot d(p, q)$.
      Sei $\varepsilon > 0$ vorgegeben.
      Setze $\delta = \varepsilon/(L+1)$.
      Dann gilt für alle $p, q \in X$ mit $d(p, q) \leq \delta$,
      dass $d(f(p), f(q)) \leq \varepsilon$.
  \end{enumerate}
\end{remarks}

\begin{lemma*}
  Seien $(X_1, d_1)$ und $(X_2, d_2)$ metrische Räume.
  Sei $f \colon X_1 \to X_2$ eine beliebige Abbildung.
  Dann ist $f$ stetig im Punkt $p$, genau dann, wenn $f$
  folgenstetig in $p$ ist.
  Folgenstetig heisst, dass für alle konvergenten Folgen
  ${(a_{n})}_{n \in \mathbb{N}}$ in $X_1$ mit
  $\lim_{n \to \infty} a_n = p$ gilt,
  dass $\lim_{n \to \infty} f(a_n) = f(p)$.
\end{lemma*}

\begin{proof}
  Für die Richtung ``$\Rightarrow$'',
  sei $f$ stetig in $p$ und ${(a_{n})}_{n \in \mathbb{N}}$
  konvergent mit Grenzwert
  $\lim_{n \to \infty} a_n = p$.
  Sei $\varepsilon > 0$ vorgegeben.
  Wähle $\delta > 0$ so, dass wenn
  $d_1(q, p) \leq \delta$ gilt,
  dann auch $d_2(f(q), f(p)) \leq \varepsilon$.
  Dann existiert $N \in \mathbb{N}$ so,
  dass für $n \geq N$ gilt, dass
  $d_1(a_n, p) \leq \delta$.
  Somit gilt auch für alle $n \geq N$,
  dass $d_2(f(a_n), f(p)) \leq \varepsilon$.
  Somit ist $f$ folgenstetig.

  Umgekehrt, für ``$\Leftarrow$'',
  sei $f$ nicht stetig in $p$.
  Dann existiert $\varepsilon > 0$ so,
  dass für alle $n \in \mathbb{N}$ ein
  $a_n \in X_1$ existiert mit
  $d_1(a_n, p) \leq 1/n$, aber
  $d_2(f(a_n), f(p)) > \varepsilon$.
  Nach Konstruktion gilt
  $\lim_{n \to \infty}a_n = p$,
  aber $\lim_{n \to \infty}f(a_n) \neq f(p)$.
  Folglich ist $f$ nicht folgenstetig in $p$.
\end{proof}

\begin{proof}[Beweis vom Fixpunktsatz]
  Wir zeigen erst die Eindeutigkeit,
  was ganz schnell geht.
  Seien $p, q \in X$ Fixpunkte von $f$.
  Dann gilt $d(f(p), f(q)) = d(p,q) \leq k \cdot d(p, q)$.
  Aus $k < 1$ folgt dann $d(p, q) = 0$,
  also $q = p$.

  Für die Existenz, wähle $a_0 \in X$. Dieser
  Schritt ist der Grund für die Annahme,
  dass $X$ nicht leer ist.
  Definiere eine Folge in $X$ durch die
  Vorschrift $a_{n+1}= f(a_n)$.
  Es gilt dann $a_1 = f(a_0), a_2 = f^2(a_0) = f(f(a_0))$,
  und allgemein $a_n = f^n(a_0)$.
  Die so konstruierte Folge ${(a_{n})}_{n \in \mathbb{N}}$
  ist eine Cauchyfolge in $X$. Wir zeigen das später
  und nehmen vorerst an, dass das stimmt.
  Da $X$ vollständig ist, existiert der Grenzwert
  $p = \lim_{n \to \infty} a_n$ in $X$.
  Es gilt dann $f(p) = p$.
  Tatsächlich, berechne
  \[
  f(p) = \lim_{n \to \infty} f(a_n) = \lim_{n \to \infty}a_{n+1} = p.
  \]
  Es bleibt also bloss zu zeigen,
  dass $ {(a_n)}_{n \in \mathbb{N}}$
  eine Cauchyfolge ist.
  Sei dazu $\varepsilon > 0$ vorgegeben.
  Für $n, m \in \mathbb{N}$ mit $n \geq m$
  schätzen wir ab, dass
  \begin{align*}
    d(a_n, a_m)
    & = d(f(a_{n-1}), f(a_{m-1}))\\
    & \leq k \cdot d(a_{n-1}, a_{m-1}) \\
    &\leq k^2 \cdot d(a_{n-2}, a_{m-2})\\
    &\leq \cdots \\
    &\leq k^m \cdot d(a_{n-m}, a_0).
  \end{align*}
  Setze $\ell = n -m $.
  Schätze weiterhin ab, dass
  \begin{align*}
    d(a_{\ell}, a_0)
    &\leq d(a_\ell, a_{\ell - 1})
    + d(a_{\ell - 1}, a_{\ell-2}) + \cdots + d(a_1, a_0)\\
    &\leq k^{\ell-1} \cdot d(a_{1}, a_{0})
    + k^{\ell - 2} \cdot d(a_1, a_0)
    + \cdots
    + d(a_1, a_0) \\
    &= (1 + k + k^2 + \cdots + k^{\ell-1})
    \cdot d(a_1, a_0) \\
    &\leq \frac{1}{1-k} \cdot d(a_1, a_0).
  \end{align*}
  Insgesamt erhalten wir
  \[
    d(a_n, a_m) \leq k^m \cdot \frac{1}{1-k} \cdot d(a_1, a_0).
  \]
  Wähle $N \in \mathbb{N}$ so,
  dass $k^N \cdot 1/(1-k) \cdot d(a_1, a_0) \leq \varepsilon$.
  Solch ein $N$ existiert, da
  nach unseren Annahmen an $k$ gilt,
  dass $\lim_{n \to \infty} k^n = 0$.
  Es gilt dann für alle $n, m \geq N$, dass
  $d(a_n, a_m) \leq \varepsilon$.
\end{proof}

Wir werden den Fixpunktsatz auf Banachräume
anwenden, um Sätze zur Existenz und Eindeutigkeit
gewisser Differentialgleichungen zu zeigen.

\begin{example}
  Sei $K \subset \mathbb{R}$ kompakt,
  das heisst, $K$ ist beschränkt und abgeschlossen.
  Definiere $C(K)$ als die Menge aller stetigen
  Abbildungen  $f \colon K \to \mathbb{R}$.
  Man kann sich leicht überlegen,
  dass $C(K)$ ein reeller Vektorraum ist.
  Wir statten $C(K)$ mit der Norm
  \[
    \Vert f \Vert_{\infty} =
    \max \left\{|f(x)| \mid x \in K\right\} < +\infty
  \]
  aus. Der Satz von Weierstrass besagt,
  dass für eine Folge $f_n \colon K \to \mathbb{R}$
  stetiger Funktionen, welche auf $K$
  gleichmässig gegen eine Funktion $f \colon K \to \mathbb{R}$
  konvergiert, $f$ stetig ist.
  Gleichmässige Konvergenz heisst,
  dass für alle $\varepsilon > 0$
  ein $N \in \mathbb{N}$ existiert, so dass
  für alle $n \geq N$ und alle $p \in K$ gilt,
  dass $|f_n(p) - f(p)| \leq \varepsilon$.
  In unseren Worten heisst das
  $\Vert f_n - f \Vert_{\infty} \leq \varepsilon$.
  Der Satz von Weierstrass besagt also,
  dass der Raum $C(K)$ mit der Maximumnorm vollständig ist.
\end{example}


\end{document}
