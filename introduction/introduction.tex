\documentclass[../main.tex]{subfiles}

\begin{document}
\section*{Über diese Vorlesung}
Diese Vorlesung besteht aus drei Teilen,
die in das Studium von Funktionen
in mehreren reellen Variablen
einführen soll. Der ganze Inhalt
dieser Vorlesung (und mehr) ist in~\cite{heuser}
zu finden,
konkret in den Kapiteln XIV, XV für
gewöhnliche Differentialgleichungen,
XX für die Differentialrechnung,
und XXI, XXIV für Gradientenfelder
und Differentialformen.

Im Kapitel~\ref{chp:odes} geht es um
``gewöhnliche Differentialgleichungen''.
Wir studieren dort Kurven
$\gamma \colon \mathbb{R} \to \mathbb{R}^n$,
deren Ableitung durch die
Position $\gamma(t)$ bestimmt ist.
Wir werden Vektorfelder einführen
um das präziser zu formulieren,
und diese dann untersuchen.
Im Kapitel~\ref{chp:differential}
geht es um differenzierbare
Ableitungen 
$f \colon \mathbb{R}^n \to \mathbb{R}^m$ 
(was sich von Kurven dadurch unterscheidet,
dass auch der Definitionsbereich
dieser Funktionen mehrdimensional sein darf).
Dieses Kapitel wird den grössten Teil
dieser Vorlesung formen.
Im Kapitel~\ref{chp:gradients} lernen
wir Gradientenfelder und Differentialformen
kennen. Dort studieren wir
Abbildungen $f \colon\mathbb{R}^n \to \mathbb{R}$.
Die leitende Frage in diesem Kapitel wird sein,
welche Vektorfelder $X \colon \mathbb{R}^n
\to \mathbb{R}$ Gradientenfelder sind.




\section*{Über dieses Dokument}
Das ist eine Mitschrift
der Vorlesung ``Analysis 2''
von Prof.\ Dr.\ Sebastian Baader
im Frühlingssemester 2021.
Du darfst sie so verwenden,
wie sie dir am meisten
beim Verständnis des Materials
hilft.
Verantwortlich dafür was
hier drin steht ist Levi Ryffel.
Denke daran dass der Dozent dieses Dokument
nicht schreibt (und vielleicht auch nicht liest).
Ihn trifft keine Verantwortung, falls
Unsinn steht.

\section*{Dein Beitrag zu den Notizen}
Diese Vorlesungsnotizen werden in Echtzeit während der Vorlesung mitgeschrieben
und werden deshalb viele Probleme enthalten.
Damit sind allerlei Missgeschicke
gemeint wie zum Beispiel
Symbolverwechslungen, unpräzise Aussagen und Argumente,
alternative Rechtschreibung und Grammatik,
oder unattraktives Layout.
Falls dir so etwas auffällt,
auch wenn es dich nicht stark stört,
und auch wenn du es
als etwas subjektiv empfindest,
poste doch auf
\[
\texttt{\href{https://github.com/raw-bacon/ana2-notes}{https://github.com/raw-bacon/ana2-notes}},
\]
ein ``Issue'', oder sende
eine E-Mail an \texttt{levi.ryffel@math.unibe.ch}.
Auf demselben Weg kannst du Wünsche
und Verbesserungsvorschläge
zu dieser Mitschrift
anbringen.


\end{document}
